\chapter{Introduction}
In this day and age, having a solid understanding of cybersecurity fundamentals is crucial in 
everyday life. These threats aren't the problem of just a few skilled people anymore, but a common crime that unfortunately 
affects hundreds of thousands of people every day.

Just to provide a glimpse of the magnitude of the problem, the Federal Bureau of Investigation's Internet Crime Complaint Center (IC3) 
releases their \href{ https://www.ic3.gov/AnnualReport/Reports/2024_IC3Report.pdf}{annual report}, in which it measures the financial losses 
due to these crimes. In 2024, the report estimated a total loss of \$16 billion dollars 
(a 33\% increase compared to 2023). These staggering losses highlight the necessity of cybersecurity awareness in every day life.

When it comes to educating users in cybersecurity, the same problem often arises: it is almost 
impossible to achieve a teaching difficulty level that suits everyone. The one-size-fit-all approach 
makes it unattractive for users to spend their time learning, as they often find the content too easy or too hard.

To help solve this problem, we used the innovative LLM technology, which enables us to personalize the user's entire experience
(from questions to personalized feedback). The main idea behind SeQG, is to take into account several factors (depending on 
the mode selected by the user), such as age and cybersecurity experience. Based on this self-assessment, the LLM will fetch 
questions according to the user's level.

In the Private Mode, in the LLM prompt we also include the user's performance on previous cybersecurity topics. This allows
the LLM to fetch more questions about the topics, where the user's knowlegde lacks, helping them make progress.

By combining all these factors, we achieve an approaching optimal learning process in a gamified way (using several question types), 
where we can adjust the difficulty level that best fits the user.
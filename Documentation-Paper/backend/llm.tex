\section{LLM - Connection}
Before the user gets to actually answer questions provided by the LLM, we need to submit our age and experience, if we did not already do so.
Only then the questions will start to get fetched from the LLM API.
But this is not happening directly, for that we also implemented a little backend, that needs to be started, which then handles all needed API calls.
Currently though the backend is not protected against calls from outside, so it is not recommended to run it on a public server.
One might first want to implement some kind of authentication, before doing so. 
The same goes for the LLM API, as it is a Ollama Backend that runns openly.
Here we also recommend to implement authentication procedures such as private keys or doing network segmentation only for the API to prevent unwanted access.

\subsection{Tips}
...

\subsection{Questions}
Here we differentiate between the two modes as one of them has a userId and the other one does not.
Considering that, lets first look at the guest-mode question api request.
For that we have prepared a prompt and a list of topics from which the llm can choose.
Both of them are stored in the backend folder.
From here on we directly pass arguments into the prompt such as age, experience and the topic.
When we are done, we simply send over the whole prompt to the LLM API, which then responds with a specific json format.
The json format is then parsed and send over to the host front-end.
Formatted the json represents this structure:
\begin{lstlisting}[language=json,firstnumber=1]
{
    "question": ... ,
    "option_s": ... ,
    "dropZones": ... ,
    "correctAnswer_s": ... ,
    "topic": ... ,
    "questionType": ...
}
\end{lstlisting}
Each one having there own meaning.
\begin{itemize}
    \item \textbf{question}: The question that the user has to answer (String).
    \item \textbf{option\_s}: The options that the user can choose from (String array).
    \item \textbf{dropZones}: Represent drop zones in a drag \& drop event, where the user can drag and drop the options (String array).
    \item \textbf{correctAnswer\_s}: Correct answers for the question (String array).
    \item \textbf{topic}: The overall topic of the question (String).
    \item \textbf{questionType}: The event-type of the question (String).
\end{itemize}

\subsection{Explanation}
...
\subsection{Saving Answers}
...
\subsection{Feedback}
...
\section{Anti-Breaching Sessions}
\section{File Cleansing}
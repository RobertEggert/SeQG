\subsection{QR Code - Integration}
QR codes play a central role in how devices connect within \textbf{SeQG}. 
They are used as a simple and intuitive way to link a user's device to an active session. 
When a session is started, the application generates a QR code containing a unique link. 
This link includes a hidden session identifier that tells the system, which session the user can connect to.
Once scanned, the device opens the link, which automatically redirects the user to the corresponding session within the app. 
The beauty of this process is that it requires almost no effort from the user, as no code or password is required. 
Everything happens in just a few seconds by scanning the QR code.

\subsection{Communication with Backend}
The \textbf{SeQG} frontend communicates with the backend through WebSockets to enable real-time, bidirectional data exchange. 
Here we differentiate between client and host frontend.
The host frontend must stay in constant connection with the backend, while a session is active, whereas the client frontend needs to connect only once. 
Before the client can connect, the host frontend must first request a JWT token and then connect to the backend with this token.
As the user joins a session, the client's frontend uses the previously fetched JWT token from the host's frontend and connects to the backend, via the URL saved in the QR Code. 
While connecting, the client's frontend emits an ''register'' event containing:
\begin{itemize}
    \item the token from the URL
    \item a role
    \item the user's unique ID from localStorage (if private-mode)
\end{itemize}
If the session is valid and the user is allowed to join, they're granted access and the interaction begins. 
If not, they receive a ''already\_connected'' message, which can then be handled as an invalid user.
This approach makes it easy to manage users in real time.